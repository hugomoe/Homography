%contient des exemples d'homographies traité par nous et ripmap

\sse{Comparison between the Ripmap and the new geometric decomposition}
%\sse{Comparaison du Ripmap et de la décomposition géométrique}

This part briefly describes the experiments shown in figures \ref{Homo1}, \ref{Homo2}, \ref{Homo3}, \ref{Homo4}, \ref{Homo5} and \ref{Homo5det}.
%Cette section décrit succintement les expériences représentées aux figures \ref{Homo1}, \ref{Homo2}, \ref{Homo3}, \ref{Homo4}, \ref{Homo5} et \ref{Homo5det}. On a choisit des homographies éloignées des cas dégénérés.


One can notice that the decompostion reduces aliasing on some homographies (homographies 1, 4 and 5). For homographies squeezing the image in a diagonal direction, the decomposition reduces the blur (homography 2). Nevertheless for some homographies both the new method and the Ripmap give similar results (homography 3). 


%On peut remarquer que la décomposition permet pour certaines homographies de limiter l'\emph{aliasing} (Homographies 1, 4 et 5). Dans le cas d'une déformation en diagonale elle limite le flou (Homographie 2). Il existe néanmoins des cas où les deux méthodes ont des performances comparables (Homographie 3).

The execution time of the program was of about one second for an image of size $512\times 512$ on a dual-core laptop computer.

%Le programme a un temps d'éxecution de l'ordre d'une seconde sur une image $512\times 512$ avec un laptop à deux processeurs.

\begin{figure}
\subfigure[Geometric decomposition]{\includegraphics[scale=0.4]{img_f_1.png}}
\subfigure[Ripmap]{\includegraphics[scale=0.4]{img_ripmap_1.png}}
\caption{Homography 1 : The geometric decomposition method creates less aliasing}
\label{Homo1}
\end{figure}

%\begin{figure}
%\subfigure[Décomposition géométrique]{\includegraphics[scale=0.4]{img_f_1.png}}
%\subfigure[Ripmap]{\includegraphics[scale=0.4]{img_ripmap_1.png}}
%\caption{Homographie 1 : La décomposition géométrique produit moins d'\emph{aliasing}}
%\label{Homo1}
%\end{figure}

\begin{figure}
\subfigure[Geometric decomposition]{\includegraphics[scale=0.4]{img_f_2.png}}
\subfigure[Ripmap]{\includegraphics[scale=0.4]{img_ripmap_2.png}}
\caption{Homography 2 : The geometric decomposition method does not create excess blur (see section \ref{Ripmap}).}
\label{Homo2}
\end{figure}



%\begin{figure}
%\subfigure[Décomposition géométrique]{\includegraphics[scale=0.4]{img_f_2.png}}
%\subfigure[Ripmap]{\includegraphics[scale=0.4]{img_ripmap_2.png}}
%\caption{Homographie 2 : La décomposition n'entraine pas d'\emph{over-blurring} (voir section \ref{Ripmap})}
%\label{Homo2}
%\end{figure}


\begin{figure}
\subfigure[Geometric decomposition]{\includegraphics[scale=0.4]{img_f_3.png}}
\subfigure[Ripmap]{\includegraphics[scale=0.4]{img_ripmap_3.png}}
\caption{Homography 3 : None of the methods seems to clearly prevail, still the decomposition introduced a bit less aliasing.}
\label{Homo3}
\end{figure}


%\begin{figure}
%\subfigure[Décomposition géométrique]{\includegraphics[scale=0.4]{img_f_3.png}}
%\subfigure[Ripmap]{\includegraphics[scale=0.4]{img_ripmap_3.png}}
%\caption{Homographie 3 : Aucune des deux méthodes n'est clairement meilleure}
%\label{Homo3}
%\end{figure}

\begin{figure}
\subfigure[Geometric decomposition]{\includegraphics[scale=0.4]{img_f_4.png}}
\subfigure[Ripmap]{\includegraphics[scale=0.4]{img_ripmap_4.png}}
\caption{Homography 4 : The geometric decomposition method creates less aliasing.}
\label{Homo4}
\end{figure}

%\begin{figure}
%\subfigure[Décomposition géométrique]{\includegraphics[scale=0.4]{img_f_4.png}}
%\subfigure[Ripmap]{\includegraphics[scale=0.4]{img_ripmap_4.png}}
%\caption{Homographie 4 : La décomposition géométrique produit moins d'\emph{aliasing}}
%\label{Homo4}
%\end{figure}



\begin{figure}
\subfigure[Geometric decomposition]{\includegraphics[scale=0.4]{img_geo_5.png}}
\subfigure[Ripmap]{\includegraphics[scale=0.4]{img_ripmap_5.png}}
\subfigure[Naive method]{\includegraphics[scale=0.4]{img_naive_5.png}}
\subfigure[Mipmap]{\includegraphics[scale=0.4]{img_mipmap_5.png}}
\caption{Homography 5 : The geometric decomposition method creates less aliasing. See the detail on figure \ref{Homo5det}.}
\label{Homo5}
\end{figure}

%\begin{figure}
%\subfigure[Décomposition géométrique]{\includegraphics[scale=0.4]{img_geo_5.png}}
%\subfigure[Ripmap]{\includegraphics[scale=0.4]{img_ripmap_5.png}}
%\subfigure[Méthode naive]{\includegraphics[scale=0.4]{img_naive_5.png}}
%\subfigure[Mipmap]{\includegraphics[scale=0.4]{img_mipmap_5.png}}
%\caption{Homographie 5 : La décomposition produit moins d'aliasing, un détail est présenté dans la figure \ref{Homo5det}}
%\label{Homo5}
%\end{figure}


\begin{figure}[t]
\centering
\subfigure[Naive method]{\includegraphics[scale=1.25]{img_det_naive.png}}\hfill
\subfigure[Mipmap]{\includegraphics[scale=1.25]{img_det_mipmap.png}}\hfill
\subfigure[Ripmap]{\includegraphics[scale=1.25]{img_det_ripmap.png}}\hfill
\subfigure[Geometric decomposition]{\includegraphics[scale=1.25]{img_det_geo.png}}
\caption{Detail of homography 5 from figure \ref{Homo5} }
\label{Homo5det}
\end{figure}



%\begin{figure}[t]
%\centering
%\subfigure[Méthode naive]{\includegraphics[scale=1.25]{img_det_naive.png}}\hfill
%\subfigure[Mipmap]{\includegraphics[scale=1.25]{img_det_mipmap.png}}\hfill
%\subfigure[Ripmap]{\includegraphics[scale=1.25]{img_det_ripmap.png}}\hfill
%\subfigure[Décomposition géométrique]{\includegraphics[scale=1.25]{img_det_geo.png}}
%\caption{Détail de l'homographie 5 qui a été présentée dans la figure \ref{Homo5} }
%\label{Homo5det}
%\end{figure}
