The new method for homography resampling is based on the decomposition of a homography interpreted as a result on an image of a motion in space of the camera. The section below presents that decomposition (theorem \ref{thepropdecomp} and corollary \ref{thecorollairedecomp}).


%La nouvelle méthode de traitement des homographies repose sur la décomposition d'une homographie qui permet d'interpréter cette dernière en terme de mouvement de caméra. La partie ci-dessous présente cette décomposition.

\ssse{Modeling camera motion}
%\ssse{Modélisation de mouvement de caméra}
\label{mouv_de_camera}

In this section we define the geometric model of a pinhole camera, and show that when its application is restricted to a plane, it can be interpreted as a planar homography. Conversely we shall see that any planar homography can be interpreted in that way. In the pinhole model, the camera is reduced to a projection from the 3D space  onto a plane trough rays passing by a focus $F$,  as illustrated in figure \ref{shmdecomp}.

%On étudie ici un cas a priori particulier d'homographie $h$ : les homographies que l'on peut interpréter comme un mouvement de caméra idéale. On montrera par la suite que c'est en fait un cas général.

%Nous modéliserons la situation en supposant que la scène filmée est plane. Cela suppose que l'on filme une surface soit sans aucun relief, soit avec un relief négligeable devant la distance à la caméra, afin qu'il ne soit pas perceptible. La figure \ref{shmdecomp} illustre la modélisation utilisée pour la caméra idéale. La caméra idéale se modélise donc par la projection d'un plan sur un autre en passant par un foyer $F$, en négligeant les lentilles ou les dispositifs correcteurs présents dans les caméras réelles.

\begin{figure}[h!]

\centering
\includegraphics[width=10cm]{shema_decomp.png}
\caption{Parameterization of a pinhole camera taking an image of the $(x,y)$ plane. $F$ is the focus of the camera, the red plane is the image plane of the camera.  One point of the image plane is obtained by the projection of a point of the plane $(x,y)$ through the point $F$. (cf. part \ref{mouv_de_camera})}
\label{shmdecomp}
\end{figure}

We consider the Euclidean space $\mathbb{R}^3$ with origin $O=(0,0,0)$ and its canonical basis denoted by $(\xbf_0,\ybf_0,\zbf_0)$.
%On se place dans l'espace vectoriel $\mathbb{R}^{3}$ on note $O$ l'origine $(0,0,0)$ et $(\xbf_0,\ybf_0,\zbf_0)$ la base canonique.
\begin{itemize}
\item Let $F$ and $C_0$ be two distinct points in $\mathbb{R}^{3}$.
\item Let $\mathcal{P}$ be the affine plane through $C_0$ and of normal vector $\overrightarrow{FC_0}$.
\item Let $\wbf$ be the vector $\frac{\overrightarrow{FC_0}}{|| \overrightarrow{FC_0}||}$
\item Let $\delta$ be the distance $|| \overrightarrow{FC_0}||$
\end{itemize}

\begin{Def}
With the above notation, we call projective mapping  and denote by $H$  the map associating with every point $X$ in $\mathbb{R}^3$ the intersection point of the straight line $(X, F)$ with the plane $\mathcal{P}$. This map has for parameters $F, \wbf$ and $\delta$.
\label{DefProjective}
%L'application projective $H$, est l'application qui à un point $X$ de $\mathbb{R}^{3}$ associe le point d'intersection entre la droite $(XF)$ et le plan $\mathcal{P}$. $H$ dépend du triplet $(F,\wbf,\delta)$.
\end{Def}

\begin{remarque}
$F$ is the focus of the camera and the plane $\mathcal{P}$ is the camera plane, on which the image contained in the plane $(x,y)$ is projected. The optical axis of the camera is the straight line $(FC_0)$ directed by $\wbf$, $\delta$ is the focal distance..
\end{remarque}

We shall denote by $\mathcal{P}'$ be the affine plane of $\mathbb{R}^{3}$ containing $F$ and parallel to the plane $\mathcal{P}$.
\begin{lem}
The projective mapping  $H$ of definition \ref{DefProjective} is defined on $\mathbb{R}^3 \setminus \mathcal{P}'$ and
\begin{equation}
H(X) = C_0 +  \delta \frac{\overrightarrow{XF}-(\overrightarrow{XF}\cdot \wbf )\wbf}{\wbf \cdot \overrightarrow{XF}}. 
\label{formul_lem_app_proj}
\end{equation}
\label{lem_app_proj}
\end{lem}

\begin{figure}[h!]

\centering
\includegraphics[width=10cm]{schema_lemme_1_bis.jpg}
\caption{Illustration of lemma \ref{lem_app_proj}}
\label{shmlemme1}
\end{figure}

%JEN SUIS LALALALALALLALALALA
\begin{proof}
If $X\in \mathbb{R}^3 \setminus \{F\}$ the straight ligne $(XF)$ and meets $\mathcal{P}$ if and only if it is not parallel to  $\mathcal{P}$ and therefore if and only if $X\in \mathbb{R}^3 \setminus \mathcal{P}'$. In that case, there exists $t_X\in \mathbb{R}$ such that
\begin{equation}
H(X)=X+t_{X}\overrightarrow{XF}.
\end{equation}
So
\begin{equation}
\overrightarrow{FH(X)} = (t_X-1)\overrightarrow{XF}
\label{formule_H}
\end{equation}
As $H(X)\in \mathcal{P}$
\begin{equation*}
\overrightarrow{FH(X)}\cdot \wbf =\delta.
\end{equation*}
Using this and \eqref{formule_H}, we obtain 
\begin{equation*}
t_{X}=1+\frac{\delta}{\wbf \cdot \overrightarrow{XF}},
\end{equation*}
 Then it can be deduced using this and \eqref{formule_H} that
 \begin{equation*}
 \overrightarrow{FH(X)}=\frac{\delta}{\wbf \cdot \overrightarrow{XF}}\overrightarrow{XF},
 \end{equation*}
 so that
 \begin{equation*}
 H(X)=C_0-\overrightarrow{FC_0}+\frac{\delta}{\wbf \cdot \overrightarrow{XF}}\overrightarrow{XF}.
 \end{equation*}
 Since $\overrightarrow{FC_0}=\delta\wbf$, it can be deduced that
\begin{equation*}
H(X) = C_0 +  \delta \frac{\overrightarrow{XF}-(\overrightarrow{XF}\cdot \wbf )\wbf}{\wbf \cdot \overrightarrow{XF}}.
\end{equation*}
\end{proof}


\begin{remarque}
The plane $\mathcal{P}'$ is the blindspot of the camera ; in the case of a real camera, that blindspot should include also the half space "behind" the camera..
\end{remarque}


\begin{Def} 
We shall denote by $P$ the canonical plane $(O,\xbf_0,\ybf_0)$.
We shall call target point the intersection point of the straight line $(FC_0)$ and of the plane $P$. This target point exists if and only if $(FC_0)$ is not parallel to $\mathcal{P}$. When it exists we will denote it by $X_v$ and we have
\begin{equation*}
X_v=F-\wbf \delta',~~~~~~\delta'=\frac{\overrightarrow{OF}\cdot \zbf_0}{\wbf \cdot \zbf_0}.
\label{formule_point_vise}
\end{equation*}
\label{point_vise}
\end{Def}


\begin{remarque}
If the target point does not exist, this means that it is at infinity and that the camera targets the horizon.
\end{remarque}


In all that follows we assume that the camera has its target point in $P$.

\begin{Def}
We call projective planar mapping $H^*$ related to $H$ the restriction of $H$ to $P\setminus (\mathcal{P}'\cap P)$. 
If $\wbf \perp P $, then $H^*$ is defined on $P$. Otherwise $H^*$ is defined on $P\setminus D$ where $D$ is the straight line
\begin{equation*}
D=\left\{ X\in P | \overrightarrow{XF}\cdot \wbf = 0\right\}.
\end{equation*}
\end{Def}

\begin{remarque}
We can interpret the situation as follows: the scene is a fully planar image $P$ photographed by the pinhole camera $H$.  Then, from a point $X\in P$ of the filmed scene, the mapping $H^*$ returns the point in $P_2$ which is its image by the camera. The straight line $D'=\{ X \in \mathcal{P} | \overrightarrow{XF} \cdot \zbf_0 =0 \}$ is called the horizon of $H^*$.
\end{remarque}

From now on we associate with the affine plane $\mathcal{P}$ a direct normalized basis $(C,\ubf,\vbf)$. 

\begin{Def}
 The homography related with $H^*$ in the basis $(C,\ubf,\vbf)$ is the mapping
\begin{equation}
h : (x,y)  \mapsto \left( \overrightarrow{CH^*(X)}\cdot \ubf , \overrightarrow{CH^*(X)}\cdot \vbf \right)
\label{formule_homographie_H}
\end{equation}
with $X=x~\xbf_0 + y~\ybf_0 $.
\label{def_homographie_H}
\end{Def}

Let $D,D'$ be the straight lines in $\mathbb{R}^2$ obtained when $P$ and $\mathcal{P}$ have the basis $(O,\xbf,\ybf)$ and $(C,\ubf,\vbf)$. Then $h:\mathbb{R}^2  \setminus D \mapsto \mathbb{R}^2  \setminus D'$ is a bijective mapping.

%\begin{remarque}
%From a point $X\in P$ of coordinates $(x,y)$, the mapping $h$ returns the coordinates of the point $H^*(X)$ of the numeric image coming from the camera. 
%\end{remarque}

\begin{Def}
\label{decompogeo_def_angles}
Let $(\phi , \theta ,\psi )$ be the three Euler angles defining the orientation of the basis $(\ubf,\vbf,\wbf)$ with respect to the basis $(\xbf_0,\ybf_0,\zbf_0)$ (figure \ref{img_angles}) :
\begin{itemize}
\item $\phi$ is the first rotation around the axis $(X_v ,\zbf_0)$
\item $\theta$ is the second rotation around the axis $(X_v , \ybf_2 )$
\item $\psi$ is the third rotation around the axis $(X_v , \wbf )$
\end{itemize}
\end{Def}

\begin{figure}
\centering
\subfigure[rotation of angle $\phi$]{\includegraphics[width=5cm]{graphe1.jpg}}
\subfigure[rotation of angle $\theta$]{\includegraphics[width=5cm]{graphe2.jpg}}
\subfigure[self rotation (of angle $\psi$)]{\includegraphics[width=5cm]{graphe3.jpg}}
\caption{Euler angles between the bases $(\ubf,\vbf,\wbf)$ and $(\xbf_0,\ybf_0,\zbf_0)$ (definition \ref{decompogeo_def_angles})}%j'ai vérifié, bases est le pluriel de basis
\label{img_angles}
\label{decompgeo_rotationPropre}
\end{figure}

In what follows we set $\cbf= \left (\overrightarrow{C_0 C}\cdot \ubf , \overrightarrow{C_0 C}\cdot \vbf \right)$ and $\xbf_v=\left( \overrightarrow{O X_v}\cdot \ubf , \overrightarrow{O X_v}\cdot \vbf \right )$.\\

 \begin{Def}
We call unidirectional homography a mapping $h:\mathbb{R}^{2} \ra \mathbb{R}^{2}$ defined by
\begin{equation*}
h(x,y)=\left ( \frac{ax+p}{rx+t} , \frac{cy+p}{rx+t} \right)
\end{equation*}
where $a,p,c,q,r,t$ are reals numbers.
\label{homo_uni_direc}
\end{Def}

\begin{prop}
Let $h$ be a planar homography as specified in Definition \ref{formule_homographie_H}. Then $h$ can be factorized as follows
\begin{equation}
h = \tau_{\cbf}   \circ R_{\psi} \circ z_{\frac{\delta}{\delta'}} \circ h_{\theta,\delta'} \circ R_{\phi} \circ \tau_{\xbf_v},
\label{formul_decomp}
\end{equation}
where $R_\alpha$ denotes the planar rotation of angle $\alpha$, $z_\lambda$ is the zoom of factor $\lambda$, $\tau_\ybf$ is the translation by the vector $-\ybf$ and $h_{\theta,\delta'}$ is the unidirectional homography (see definition \ref{homo_uni_direc}) defined by
\begin{equation}
h_{\theta,\delta'}(x,y)=\left(\frac{-cos(\theta)x}{1-\frac{sin(\theta)}{\delta'}x} ,\frac{-y}{1-\frac{sin(\theta)}{\delta'}x}\right).
\label{mise_perspective}
\end{equation}
\label{prop_decomp}
\end{prop}








\begin{proof}
Using \eqref{formule_homographie_H}, definition \ref{def_homographie_H} and \eqref{formul_lem_app_proj} of lemma \ref{lem_app_proj}, we have
\begin{equation*}
h((x,y))=\left( \delta \frac{\overrightarrow{XF}\cdot \ubf}{\overrightarrow{XF} \cdot \wbf} +\overrightarrow{CC_0 } \cdot \ubf, \delta \frac{\overrightarrow{XF}\cdot \vbf}{\overrightarrow{XF} \cdot  \wbf}+\overrightarrow{CC_0 }\cdot \vbf \right).
\end{equation*}
where $X=x \xbf_0 + y \ybf_0 $. By introducing the target point $X_v$ (see \eqref{formule_point_vise} of definition \ref{point_vise}) and the translation $\tau_c$ we have
\begin{equation}
(\tau_\cbf^{-1} \circ h)((x,y)) = \left ( -\delta \frac{\overrightarrow{X_vX}\cdot \ubf}{\delta' -\overrightarrow{X_vX} \cdot \wbf},-\delta \frac{\overrightarrow{X_vX}\cdot \vbf}{\delta' -\overrightarrow{X_vX} \cdot \wbf} \right).
\label{decomp_formul_intermediaire_1}
\end{equation}

\noindent We define an intermediate basis $(\xbf_1 ,\ybf_1 ,\zbf_1)$  $(\xbf_2 ,\ybf_2 ,\zbf_2)$, in order to decompose the three rotations $\phi,\theta,\psi$ (figures \ref{img_angles} and \ref{decompgeo_rotationPropre} ). Then we have
\begin{equation*}
\ubf=\cos(\psi)\xbf_{2}+\sin(\psi)\ybf_{2} ,  \vbf=-\sin(\psi)\xbf_{2}+\cos(\psi)\ybf_{2} \text{ and } \wbf=\zbf_2.
\end{equation*}

\noindent Denoting by $R_{s}$ the rotation of angle $s$ and using  \eqref{decomp_formul_intermediaire_1}, we obtain
\begin{equation*}
(\tau_{\cbf}^{-1} \circ h)((x,y)) = R_{\psi}\left(\frac{\delta \overrightarrow{X_v X}\cdot \xbf_{2} }{\delta'-\zbf_2 \cdot \overrightarrow{X_v X}},\frac{\delta \overrightarrow{X_v X}\cdot \ybf_{2}}{\delta'-\zbf_2 \cdot \overrightarrow{X_v X}}  \right).
\end{equation*}
Let $i$ be the mapping such that $i(x,y)=x \xbf_0 + y \ybf_0$. Since $i(\xbf_v)=\overrightarrow{O X_v}$ we have
\begin{equation*}
(R_{\psi}^{-1} \circ \tau_{\cbf}^{-1}  \circ h)((x,y))=\delta \left(\frac{-i(\tau_{\xbf_v} ((x,y)))\cdot \xbf_{2} }{\delta'-\zbf_2 \cdot i(\tau_{\xbf_v} ((x,y)))},\frac{-i(\tau_{\xbf_v} ((x,y)))\cdot \ybf_{2}}{\delta'-\zbf_2 \cdot i(\tau_{\xbf_v} ((x,y)))}  \right) 
\end{equation*}
Since $\zbf_{2}=cos(\theta)\zbf_{1}+sin(\theta)\xbf_{1}$, $\xbf_{2}=cos(\theta)\xbf_{1}-sin(\theta)\zbf_{1}$ (figure \ref{img_angles}) and $\zbf_{1}\perp P_{1}$, we have
\begin{equation*}
(R_{\psi}^{-1} \circ \tau_{\cbf}^{-1}  \circ h)((x,y))=\frac{\delta}{\delta'}\left(\frac{-\cos(\theta)i(\tau_{\xbf_v} ((x,y)))\cdot \xbf_{1} }{1-\frac{sin(\theta)}{\delta'}\xbf_{1}\cdot i(\tau_{\xbf_v}((x,y)))}, \frac{-i(\tau_{\xbf_v} ((x,y)))\cdot \ybf_{1}}{1-\frac{sin(\theta)}{\delta'}\xbf_{1}\cdot i(\tau_{\xbf_v}((x,y)))}  \right) 
\end{equation*}
Let $h_{\theta,\delta'}$ be defined by
\begin{equation*}
h_{\theta,\delta'}(x',y')=\left(\frac{-\cos(\theta)x'}{1-\frac{\sin(\theta)}{\delta'}x'} ,\frac{-y'}{1-\frac{\sin(\theta)}{\delta'}x'}\right)
\end{equation*}
Then
\begin{equation*}
(R_{\psi}^{-1} \circ \tau_{\cbf}^{-1} \circ h)((x,y))= \frac{\delta}{\delta'}h_{\theta,\delta'}\left ( i(\tau_{\xbf_v}((x,y))) \cdot \xbf_{1}, i(\tau_{\xbf_v}((x,y))) \cdot \ybf_{1}\right).
\end{equation*}
\label{figure_de_rotations_18}
Since $\xbf_{1}=\cos(\phi)\xbf_{0}+\sin(\phi)\ybf_{0}$ et $\ybf_{1}=-\sin(\phi)\xbf_{0}+\cos(\phi)\ybf_{0}$ (figure \ref{img_angles}), we have

\begin{eqnarray*}
(R_{\psi}^{-1} \circ \tau_{\cbf}^{-1} \circ h)((x,y)) &=& \frac{\delta}{\delta'}h_{\theta,\delta'}\left ( R_{\phi}(i(\tau_{\xbf_v}((x,y))) \cdot \xbf_{0}, i(\tau_{\xbf_v}((x,y))) \cdot \ybf_{0})\right)\\
                                               &=&\frac{\delta}{\delta'} (h_{\theta,\delta'}\circ R_{\phi} \circ \tau_{\xbf_v})((x,y)).
\end{eqnarray*}
Denoting zooms $z_{\lambda}:X\rightarrow \lambda X$, the claim is verified since
\begin{equation*}
h = \tau_{\cbf} \circ R_{\psi} \circ z_{\frac{\delta}{\delta'}} \circ h_{\theta,\delta'} \circ R_{\phi} \circ \tau_{\xbf_v}
\end{equation*}
\end{proof}


\begin{remarque}
Resampling an image by the homography $h$ simulates a change of the point of view on the input image. This change of viewpoint has parameters $(\phi,\theta,\psi,\delta,\delta',\xbf_v,\cbf_v)$, which are not independent. %The case in which the camera is targeting the horizon has not been discussed, translations $\tau_\cbf$ allow to avoid this.
\end{remarque}

\begin{remarque}
The previous method does not model every affine transform. Indeed the function $h$ defined by 
\begin{equation*}
h = \tau_{\cbf}   \circ R_{\psi} \circ z_{\frac{\delta}{\delta'}} \circ h_{\theta,\delta'} \circ R_{\phi} \circ \tau_{\xbf_v}
\end{equation*}
is affine if and only if $\theta=0$. In that case we have
\begin{equation*}
h= \tau_{\cbf} \circ z_{-\frac{\delta}{\delta'}} \circ R_{\phi+\psi} \circ \tau_{\xbf_{v}}
=\tau' \circ z_{-\frac{\delta}{\delta'}} \circ  R_{\phi+\psi}.
\end{equation*}
However the affinity can be seen as a limit case. Let $k$ be the ratio $\frac{\delta}{\delta'}$. If $\delta'$ and $\delta$ tend to $+\infty$, the function $h_{\theta,\delta'}$ tends to $h_{\theta,\infty}$ defined by
\begin{equation*}
h_{\theta,\infty}=(x,y)=(-\cos(\theta)x,-y).
\end{equation*}
Physically, this process is equivalent to moving away from the plane while increasing the focal distance in order to keep the size of the output image constant.

\noindent If $h_\infty = z_{-\frac{\delta}{\delta'}} \circ \tau_{\xbf} \circ R_{\phi} \circ h_{\theta,\infty} \circ R_{\psi} \circ \tau_{\xbf_{v}}$, the linear part of $h_{\infty}$ can be represented by a matrix $2\times2$

\begin{equation*}
R_{\psi} \cdot 
\begin{pmatrix}
-k\cos(\theta)&0\\
0&-k
\end{pmatrix}
\cdot R_{\phi}
\end{equation*}

\begin{lem}
Let $M$ be an invertible $2\times 2$ matrix. Then (by the singular value decomposition \cite{morel2009asift}) there exist two rotation matrix $R_1$ and $R_2$ and a diagonal matrix $D$ such that $M = R_1 \cdot D \cdot R_2$.
\label{decomp_valeur_sing}
\end{lem}

\noindent Using lemma \ref{decomp_valeur_sing}, it can be deduced that for every bijective affinity $A$, there exists a camera movement $h$ such that $h_\infty = A$. Moreover it can be assumed that $h$ has no output translation.
\end{remarque}

\subsubsection{Application of the decomposition to homographies}
The previous results show that all homographies can be decomposed as described in the next theorem.

%Le résultat précédent montre que certaine homographie peuvent se décomposer de la
\begin{thm}
Let $h$ be a planar homography. If $h$ is not an affine map, then there exist parameters $(\phi,\theta,\psi,\delta,\delta',(x_1,y_1),(x_2,y_2))$ such that
\begin{equation*}
h = \tau_{(x_2,y_2)} \circ R_{\psi} \circ z_{\frac{\delta}{\delta'}} \circ h_{\theta,\delta'} \circ R_{\phi} \circ \tau_{(-x_1,-y_1)}
\end{equation*}
This decomposition is not unique. More precisely, for all $\lambda \in ]0,1[$,

  \begin{equation*}
h = \tau_{(x_2,y_2)} \circ z_{\frac{\delta}{\delta'}}  \circ R_{\psi} \circ h_{\theta,\delta'} \circ R_{\phi} \circ \tau_{(-x_1,-y_1)}
  \end{equation*}
  where 
 \begin{equation*}
x_2=\frac{ar+sb+\hat r \lambda}{r^2 +s^2}, y_2=\frac{cr+sd+\hat s \lambda}{r^2 +s^2}, (x_1 , y_1) = h^{-1}(x_{2},y_{2})
  \end{equation*}
 \begin{equation*}
 \cos( \phi )= - \frac{r}{\sqrt{r^2 + s^2}}, \sin( \phi )= - \frac{s}{\sqrt{r^2 + s^2}},\cos( \psi ) =- \frac{\hat r}{\sqrt{\hat r^2 + \hat s^2}}, \sin( \psi ) = \frac{\hat s}{\sqrt{\hat r^2 + \hat s^2}}
 \end{equation*}
 \begin{equation*}
 \frac{\delta}{\delta'}=|\lambda|\left(\frac{\hat r^2 + \hat s^2}{r^2 + s^2}\right)^{3/2}, \cos(\theta)=\lambda, \sin(\theta)=\sqrt{1-\lambda^2}, \delta'=  \frac{\sqrt{(r^2 + s^2)(1-\lambda^2)}}{|\lambda| (\hat r^2+\hat s^2)}
 \end{equation*}
\label{thepropdecomp}
\end{thm}

\begin{corollaire} If $h$ is a planar homography and $h$ is not affine, then there exists a translation $\tau$, two rotations $R_\phi ,R_\psi$ and a unidirectional homography $\tilde{h}$ such that
\begin{equation}
h=\tau \circ R_\psi \circ \tilde{h} \circ R_\phi,
\label{formule_decomposition_effective}
\end{equation}
yet this decomposition is not unique.
\label{thecorollairedecomp}
\end{corollaire}

\begin{remarque}
The formula \eqref{formule_decomposition_effective} is the main ingredient of algorithm \ref{pseudoCodeDecompo}.
\end{remarque}


\begin{proof}
	 Using theorem \ref{thepropdecomp}, there exist $(\phi,\theta,\psi,\delta,\delta',\xbf_v,\cbf)$ such that 
	 \begin{equation*}
	 h = \tau_{\cbf} \circ R_{\psi} \circ z_{\frac{\delta}{\delta'}} \circ h_{\theta,\delta'} \circ R_{\phi} \circ \tau_{\xbf_v}.
	 \end{equation*}
	 
\noindent Observe that $R_\psi$ and $z_{\frac{\delta}{\delta'}}$ commute. Let $\tau'$ be the translation such that $\tau' \circ R_\phi =  R_\phi \circ \tau_{\xbf_v}$.\\
	 Then setting $\tilde{h} = z_{\frac{\delta}{\delta'}} \circ h_{\theta,\delta'} \circ \tau'$ it can be easily verified that $\tilde{h}$ is a unidirectional homography.
	 \end{proof}
	\label{ref_schema_decomp_cool}
	\begin{figure}
		\centering
		\subfigure[Input image]{
		\centering
		{\includegraphics[scale=0.24]{vue_fps_identity.png}}
		{\includegraphics[scale=0.35]{vue_tps_identity.png}}}
		\subfigure[After a first rotation (of angle $\phi$)]{
		\centering
		{\includegraphics[scale=0.24]{vue_fps_rotation_phi.png}}
		{\includegraphics[scale=0.35]{vue_tps_rotation_phi.png}}}
		\subfigure[After the unidirectional homography]{
		\centering
		{\includegraphics[scale=0.24]{vue_fps_hom_part.png}}
		{\includegraphics[scale=0.35]{vue_tps_hom_part.png}}}
		\subfigure[Output image (after the rotation of angle  $\psi$)]{
		\centering
		{\includegraphics[scale=0.24]{vue_fps_rotation_psi.png}}
		{\includegraphics[scale=0.35]{vue_tps_rotation_psi.png}}}
		\caption{Steps to process a homography, represented as camera motion. On the left, the view from the camera, on the right, a motionless representation of the scene. The translations are not represented to simplify the situation. In the motionless views, $F$ is the focus of the camera, the red plane is the image plane of the camera (see section \ref{ref_schema_decomp_cool})}.
		\label{schema_decomp_cool}
		\label{SchemaEtapesDecompoGeometrique}
	\end{figure}
	\clearpage
